\documentclass[../main.tex]{memoir}

\begin{document}
\thispagestyle{empty}

\begin{center}
  {\large\bfseries \ProjectTitle}\\
\end{center}

\begin{center}
  \AuthorName\\
  \vspace{0.7cm}
  \noindent{\textbf{Keywords}: reverse mathematics, axioms, computability theory, arithmetical hierarchy, logical equivalence.}\\
  \vspace{0.7cm}
  \noindent{\textbf{Overview}}\\
\end{center}

The field of reverse mathematics as we know it today began in 1975 with Harvey Friedman's seminal paper ``Some Systems of Second Order Arithmetic and Their Use'' \cite{friedman}. In the following years it became more of a program with a lot of research effort directed towards the development, improvement and settlement of its defining foundations and characteristic, as well as further study of its consequences and more interesting results. In 2009, Stephen G. Simpson published his encyclopaedic treatise on the current state of the field, ``Subsystems of Second Order Arithmetic'' \cite{simpson}, which has been extensively referenced and consulted for the creation of the present document. A friendlier introduction to some of the most relevant topics covered in Simpson's book can be found in John Stillwell's ``Reverse Mathematics: Proofs from the Inside Out'' \cite{stillwell}, which has also served as a source of inspiration for the more didactical parts of this work.

As it stands, reverse mathematics is concerned with the relation between axioms and theorems. For most of the history of mathematics, axioms have been regarded as a foundational tool from which to build the rest of the mathematical results in a deductive process with which mathematicians are nowadays completely familiar. Even though there exist some historically important examples of work in the study of axioms by themselves, most notably Euclid's fifth postulate and the axiom of choice or the continuum hypothesis for set theory, the current field of reverse mathematics was very much unexplored until Friedman's introduction of the tools and ideas required to accomplish its goals. In its most basic form, the key idea of reverse mathematics is that for certain axioms and theorems, their relation is not only of implication but also of equivalence. That is, in Harvey Friedman's own words: ``When the theorem is proved from the right axioms, the axioms can be proved from the theorem''. Of course, this idea needs much clarification and formalization in order to be useful for mathematicians, and this is precisely what the bulk of our effort will be in the following chapters.

The introduction chapter lays out the basic ideas and concepts with which we will work in order to reach our ultimate goal. The most important principle outlined here is the formalization of what is to be understood when we say some axiom is equivalent to some theorem. For that, we will need a base axiomatic system in which to carry out the proof of equivalence. The classic approach would be to assume the base system \textbf{and} the target axiom to show that the theorem holds. Reverse mathematics gets its name from the other part of the proof: assuming the base system and the theorem and proving the axiom. It is important to note here that the base system must be powerful enough to be able to express the axiom and the theorem (e.g., if we were to prove the Weierstra{\ss} theorem we first need to define what a continuous function is in our base system), but also weak enough so that neither the target axiom nor the theorem follow from it.

Having presented the fundamental ideas and definitions, we proceed to some more advanced and detailed concepts. First and foremost we make some necessary definitions to ultimately arrive at the concept of the arithmetical hierarchy, which will be capital for our upcoming work, by which we can give a sense of \textit{complexity} to some second order formulas of our language, i.e., how much \textit{complex} is a certain formula. We also define the system of second order arithmetic, and the subsystem we will be most interested in, namely \rca. Finally, we show how the pairing function is enough to \textit{embed} the integer and rational numbers inside the natural numbers, and present a syntetic but useful definition of real numbers within \rca.

The following chapter is concerned with computability theory, which may look like a totally unrelated topic at first. We first present the classical approach of recursion theory, in which computability is inductively defined for a reduced class of initial functions and certain allowed \textit{combinations} of them. We also introduce the more well-known Turing machines approach and outline a sketch of a proof for their equivalence (in that they prove the exact same class of functions are \textit{computable}). We end this section by stating the Church-Turing thesis, which allows us to proceed more loosely in the rest of the chapter, but with confidence that our results can be backed up by actually rigorous arguments should we need them.

We then explore the concept of recursive enumerability and present some examples and properties, only to discover that we can actually enumerate Turing machines (or partial recursive functions, for that matter) and even show that there exists a \textbf{universal Turing machine}. This way we find a very natural example of a r.e. set that is not computable. In fact, this concrete example is very intimately related to the halting problem, which serves us to tie our theoretical developments with some of the most renowned aspects of computability theory.

Later, we introduce the concept of reduction, particularized to the cases of many-one reductions and Turing reductions. This allows us to explore oracle Turing machines, the generalization of our previous work in the form of \textbf{relativized computability} and Turing degrees. In the end, all of the theoretical aspects of the computability theory developed up until this point allow us to state Post's theorem, which deeply relates the concepts studied in this chapter with the arithmetical hierarchy. In particular, roughly speaking, it turns out that arithmetical hierarchy is to Turing degrees as \rca\ is to computability (or computable objects). In other words, \rec\ is the same as computable and \re\ is equivalent to recursive enumerability.

After that we present again the \rca\ system and some of the constructions we will be most interested in. Specifically, we define in \rca\ sequences of rational numbers, sequences of real numbers and continuous functions. In this chapter two more subsystems of second order arithmetic are also introduced: \wkl\ and \aca.

For \wkl, we define in \rca, as usual, what are finite sets, finite sequences, binary trees and infinite paths in trees. Then we present weak König's lemma, and define the system \wkl\ comprising the axioms of \rca\ plus weak König's lemma. Finally, arithmetical formulas and the full second order axiom scheme of induction were already introduced in the first chapter so the system of \aca\ consisting in the axioms of \rca\ together with second order induction and arithmetical comprehension is very easily defined. In this chapter we also explore some other aspects of these subsystems such as the structure of their $\omega$-models in terms of computability theory and their first order parts, i.e., what first order formulas can they prove.

Lastly, the final chapter accomplishes the goal we set for ourselves at the beginning of the work and succesfully uses all the theory developed to showcase some concrete examples of reverse mathematics. In particular, we prove that \wkl\ and \aca\ are each equivalent to a number of classical results in analysis over \rca, such as the Heine-Borel or the Bolzano-Weierstra{\ss} theorems. In doing so, we show how reverse mathematics can be used as an effective tool for determining the relative strength of theorems when compared. For example, the fact that the monotone convergence theorem is equivalent to \aca\ and \aca\ stronger than \wkl\ means that it is \textit{stronger} than the compactness of the interval $[0, 1]$, otherwise referred to as the Heine-Borel theorem.

An additional chapter for conclusions and further work pinpoints the key ideas and results developed in the preceding chapters and shows how there are plenty of related topics not covered in this document.

\newpage
\end{document}

%%% Local Variables:
%%% mode: latex
%%% TeX-master: "../main"
%%% End:
